%%%%%%%%%%%%%%%%%%%%%%%%%%%%%%%%%%%%%%%%%
% Beamer Presentation
% LaTeX Template
% Version 1.0 (10/11/12)
%
% This template has been downloaded from:
% http://www.LaTeXTemplates.com
%
% License:
% CC BY-NC-SA 3.0 (http://creativecommons.org/licenses/by-nc-sa/3.0/)
%
%%%%%%%%%%%%%%%%%%%%%%%%%%%%%%%%%%%%%%%%%

%----------------------------------------------------------------------------------------
%	PACKAGES AND THEMES
%----------------------------------------------------------------------------------------

\documentclass{beamer}

\mode<presentation> {

% The Beamer class comes with a number of default slide themes
% which change the colors and layouts of slides. Below this is a list
% of all the themes, uncomment each in turn to see what they look like.

%\usetheme{default}
%\usetheme{AnnArbor}
%\usetheme{Antibes}
%\usetheme{Bergen}
%\usetheme{Berkeley}
%\usetheme{Berlin}
%\usetheme{Boadilla}
%\usetheme{CambridgeUS}
%\usetheme{Copenhagen}
%\usetheme{Darmstadt}
%\usetheme{Dresden}
%\usetheme{Frankfurt}
%\usetheme{Goettingen}
%\usetheme{Hannover}
%\usetheme{Ilmenau}
%\usetheme{JuanLesPins}
%\usetheme{Luebeck}
\usetheme{Madrid}
%\usetheme{Malmoe}
%\usetheme{Marburg}
%\usetheme{Montpellier}
%\usetheme{PaloAlto}
%\usetheme{Pittsburgh}
%\usetheme{Rochester}
%\usetheme{Singapore}
%\usetheme{Szeged}
%\usetheme{Warsaw}

% As well as themes, the Beamer class has a number of color themes
% for any slide theme. Uncomment each of these in turn to see how it
% changes the colors of your current slide theme.

%\usecolortheme{albatross}
%\usecolortheme{beaver}
%\usecolortheme{beetle}
%\usecolortheme{crane}
%\usecolortheme{dolphin}
%\usecolortheme{dove}
%\usecolortheme{fly}
%\usecolortheme{lily}
%\usecolortheme{orchid}
%\usecolortheme{rose}
%\usecolortheme{seagull}
%\usecolortheme{seahorse}
%\usecolortheme{whale}
%\usecolortheme{wolverine}

%\setbeamertemplate{footline} % To remove the footer line in all slides uncomment this line
%\setbeamertemplate{footline}[page number] % To replace the footer line in all slides with a simple slide count uncomment this line

%\setbeamertemplate{navigation symbols}{} % To remove the navigation symbols from the bottom of all slides uncomment this line
}

\usepackage{graphicx} % Allows including images
\usepackage{booktabs} % Allows the use of \toprule, \midrule and \bottomrule in tables

%----------------------------------------------------------------------------------------
%	TITLE PAGE
%----------------------------------------------------------------------------------------

\title[Signal Processing on Graphs]{Journal Club: The Emerging Field of Signal Processing on Graphs} % The short title appears at the bottom of every slide, the full title is only on the title page

\author{Matt Bartos} % Your name
\institute[UM] % Your institution as it will appear on the bottom of every slide, may be shorthand to save space
{
University of Michigan \\ % Your institution for the title page
\medskip
\textit{mdbartos@umich.edu} % Your email address
}
\date{\today} % Date, can be changed to a custom date

\begin{document}

\begin{frame}
\titlepage % Print the title page as the first slide
\end{frame}

\begin{frame}
\frametitle{Overview} % Table of contents slide, comment this block out to remove it
\tableofcontents % Throughout your presentation, if you choose to use \section{} and \subsection{} commands, these will automatically be printed on this slide as an overview of your presentation
\end{frame}

%----------------------------------------------------------------------------------------
%	PRESENTATION SLIDES
%----------------------------------------------------------------------------------------

%------------------------------------------------
\section{Other work on graph signals} % Sections can be created in order to organize your presentation into discrete blocks, all sections and subsections are automatically printed in the table of contents as an overview of the talk
%------------------------------------------------

\begin{frame}
\frametitle{Other work on graph signals}

\begin{description}
\item[ZFC] Zhang, C., Florencio, D., \& Chou, P. A. (2015). Graph signal processing: a
  probabilistic framework. Microsoft Res., Redmond, WA, USA, Tech. Rep.
  MSR-TR-2015-31.

\item[MSLR] Marques, A. G., Segarra, S., Leus, G., \& Ribeiro, A. (2016). Sampling of
  graph signals with successive local aggregations. IEEE Transactions on Signal
  Processing, 64(7), 1832-1843.

\item[PV] Perraudin, N., \& Vandergheynst, P. (2017). Stationary signal processing on
  graphs. IEEE Transactions on Signal Processing, 65(13), 3462-3477.
\end{description}
\end{frame}

%------------------------------------------------
\section{Contribution of Shuman et al.}
%------------------------------------------------

\begin{frame}
  \frametitle{Contribution of Shuman et al.}
  \begin{itemize}
  \item This is a tutorial review paper
  \end{itemize}
\end{frame}

\begin{frame}
  \frametitle{Challenges of graph signal processing}
  How can we extend traditional signal processing tools to graphs?
  \begin{itemize}
    \item Translation, downsampling and modulation of signals in the graph
      domain
    \item How to implement filtering operations on graphs
  \end{itemize}
\end{frame}

%------------------------------------------------

\begin{frame}
\frametitle{Review of the Fourier Transform}

In the traditional Euclidean domain, we can find the frequency-domain
representation of a time series signal using the Fourier transform:

\begin{block}{Classical Fourier Transform}
  \begin{equation}
    \hat{f}(\xi) = \int_{-\infty}^{\infty} f(t) e^{-2 \pi j \xi t} dt 
  \end{equation}
\end{block}

$\hat{f}$ is an expansion of $f$ in terms of complex exponentials, which are the
eigenfunctions of the Laplace operator:

\begin{block}{Classical Fourier transform in terms of the Laplace operator}
  \begin{equation}
    - \Delta (e^{2 \pi j \xi t}) = \frac{\partial^2}{\partial t^2} e^{2 \pi j \xi t} = 4 \pi^2 \xi^2 e^{2 \pi j \xi t} = k e^{2 \pi j \xi t}
  \end{equation}
\end{block}
\end{frame}


\begin{frame}
\frametitle{The Graph Laplacian}

For each vertex $i$, the Laplace operator computes the weighted sum of the
differences between the signal value at $i$ and the signal value at $i's$
neighbors ($j \in N_i$). 

\begin{block}{The Laplacian is a difference operator}
  \begin{equation}
    (\Delta f) (i) = (L f) (i) = \sum_{j \in N_i} W_{i, j} [f(i) - f(j)]
  \end{equation}
\end{block}

The Laplace operator for an undirected graph is simply the degree matrix minus the
adjacency matrix.

\begin{block}{The Graph Laplacian}
  \begin{equation}
    L = D - W
  \end{equation}
\end{block}

$L$ will have a full set of orthonormal eigenvectors, and real eigenvalues. Zero
will occur as an eigenvalue with multiplicity equal to the number of connected
components.
\end{frame}

\begin{frame}
  \frametitle{The Graph Fourier Transform}
  The Graph Fourier transform is an expansion of $f$ in terms of the
  eigenvectors $u_l$ of the Graph Laplacian.
  
  \begin{block}{The Graph Fourier Transform}
    \begin{equation}
      \hat{f}(\lambda_l) = \sum_{i=1}^N f(i) u^*_l(i) 
    \end{equation}
  \end{block}

  \begin{itemize}
    \item Eigenvectors associated with the smallest eigenvalues vary slowly across
      the graph.
    \item Eigenvectors associated with the largest eigenvalues oscillate rapidly.
  \end{itemize}
\end{frame}

\begin{frame}
  \frametitle{Measuring local signal smoothness on a graph}
  \begin{block}{The edge derivative of $f$ with respect to edge $e=(i,j)$}
    \begin{equation}
      \frac{\partial f}{\partial e} \bigg|_i = \sqrt{W_{i,j}} [f(j) - f(i)]
    \end{equation}
  \end{block}

  The local variation can be measured by the square root of the sum of the squared differences
  between signal values at adjacent vertices.
  
  \begin{block}{The local variation at vertex $i$}
    \begin{equation}
      || \nabla_i f || = \bigg[ \sum_{\text{e connected to i}} \bigg( \frac{\partial f}{\partial e} \bigg|_i \bigg)^2 \bigg]^{1/2} = \bigg[ \sum_{j \in N_i} W_{i,j} [f(j) - f(i)]^2 \bigg]^{1/2}
    \end{equation}
  \end{block}
\end{frame}

\begin{frame}
  \frametitle{Measuring global signal smoothness on a graph}

  \begin{block}{Discrete p-Dirichlet form of $f$}
    \begin{equation}
      S_p(f) = \frac{1}{p} \sum_{i \in V} \bigg[ \sum_{j \in N_i} W_{i,j} [f(j) - f(i)]^2 \bigg]^{\frac{p}{2}}
    \end{equation}
  \end{block}

  For $p=1$, $S_1$ is simply the sum of local variations across all vertices.

  For $p=2$, $S_2$ is a quadratic function of the Laplacian:

  \begin{block}{Graph Laplacian Quadratic Form}
    \begin{equation}
      \begin{split}
        S_2(f) = \frac{1}{2} \sum_{i \in V} \bigg[ \sum_{j \in N_i} W_{i,j}
        [f(j) - f(i)]^2 \bigg]^{\frac{1}{2}} = \sum_{(i, j) \in \epsilon}
        W_{i,j} [f(j) - f(i)]^2 \\
        = f^T L f
      \end{split}
    \end{equation}
  \end{block}

$S_2$ is small when $f$ has similar values at strongly-connected vertices.

\end{frame}

\begin{frame}
  \frametitle{Alternatives to the Graph Laplacian}

  \begin{block}{Normalized Graph Laplacian}
    \begin{equation}
      \tilde{L} = D^{-1/2} L D^{-1/2}
    \end{equation}
  \end{block}

  The eigenvalues of $\tilde{L}$ will be between 0 and 2. For bipartite graphs,
  the spectral folding phenomenon can be used.

  \begin{block}{Random Walk Matrix}
    \begin{equation}
      P = D^{-1} W
    \end{equation}
  \end{block}

  \begin{block}{Asymmetric Graph Laplacian}
    \begin{equation}
      L_a = I - D^{-1} W
    \end{equation}
  \end{block}
  
\end{frame}

\begin{frame}
  \frametitle{Filtering in the frequency/graph spectral domain}

  Using some transfer function $\hat{h}$, we can filter an input signal as follows:
  \begin{block}{Classical frequency filtering}
    \begin{equation}
      f_{out}(t) = \mathcal{F}^{-1} \{ \hat{f}_{in}(\xi) \hat{h}(\xi)\}
    \end{equation}
  \end{block}

  In the graph setting:
  
  \begin{block}{Graph filtering in the graph spectral domain}
    \begin{equation}
      \begin{split}
        \hat{h}(L) = U
        \begin{bmatrix}
          \hat{h}(\lambda_0) & & 0 \\
           & \ddots & \\
          0 & & \hat{h}(\lambda_{N-1})
        \end{bmatrix}
        U^*
      \\\\
      f_{out} = \hat{h}(L) f_{in}
      \end{split}
    \end{equation}
  \end{block}
  
\end{frame}

\begin{frame}
  \frametitle{Filtering in the time/vertex domain}

  We can also filter in the time domain using convolution:

  \begin{block}{Classical time-domain filtering}
    \begin{equation}
      f_{out}(t) = (f_{in} * h)(t) 
    \end{equation}
  \end{block}

  In the graph setting, the output at any vertex $i$ is a linear combination of the elements of
  the input signal within a K-hop neighborhood (for some constants $b$):
  
  \begin{block}{Graph filtering in the vertex domain}
    \begin{equation}
      f_{out}(i) = b_{i,i} f_{in}(i) + \sum_{j \in N(i, K)} b_{i,j} f_{in}(j)
    \end{equation}
  \end{block}  
\end{frame}

\begin{frame}
  \frametitle{Equivalence of vertex/spectral filtering}
  If the frequency filter is a K-order polynomial $\hat{h} = \sum_{k=0}^K a_k
  \lambda_l^k$, the frequency filtered signal at vertex $i$ is a linear combination of the
  elements of the input signal at vertices within a K-hop neighborhood:

  \begin{block}{Frequency filtering when the filter is a K-order polynomial}
    \begin{equation}
      f_{out}(i) = b_{i,i} f_{in}(i) + \sum_{j \in N(i, K)} \sum_{d G(i,j)}^K a_k (L^k)_{i,j} f_{in}(j)
    \end{equation}
  \end{block}  
\end{frame}

\begin{frame}
  \frametitle{Convolution}

  Although we cannot directly generalize a convolution product on a graph
  because $h(t - \tau)$ is undefined, we can use frequency filtering, as
  previously defined:
  
  \begin{block}{Convolution of a signal on a graph}
    \begin{equation}
      (f * h)(i) = \sum_{l=0}^{N-1} \hat{f}(\lambda_l) \hat{h}(\lambda_l) u_l(i)
    \end{equation}
  \end{block}  
\end{frame}

\begin{frame}
  \frametitle{Translation}

  \begin{block}{Classical translation operation}
    \begin{equation}
      (T_vf)(t) = f(t - v) = (f * \delta_v)(t)
    \end{equation}
  \end{block}  

  Again, we cannot directly generalize $(t - v)$ for a graph, so we consider
  instead the definition of translation as convolution with a Dirac delta. 
  
  \begin{block}{Translation of a graph signal}
    \begin{equation}
     (T_n f)(i) = \sqrt{N} (f * \delta_n)(i) = \sqrt{N} \sum_{l=0}^{N-1} \hat{f}(\lambda_l) u_l^*(n) u_l(i)
    \end{equation}
  \end{block}  

Where:

\begin{equation}
  \delta_n =
  \begin{cases}
    1 \ if \ i=n \\
    0 \ otherwise
  \end{cases}
\end{equation}

\end{frame}

\begin{frame}
  \frametitle{Modulation}

  In simple terms, like a ``translation'' in the frequency domain.
  
  \begin{block}{Classical modulation}
    \begin{equation}
      \begin{split}
        \text{Time domain: } (M_\omega f)(t) = e^{2 \pi j \omega t} f(t) \\
        \text{Frequency domain: } \overline{M_\omega f}(\xi) = \hat{f}(\xi -
        \omega)
      \end{split}
    \end{equation}
  \end{block}  

  Replace complex exponential with a graph Laplacian eigenvector:
  
  \begin{block}{Graph modulation}
    \begin{equation}
      (M_kf)(i) = \sqrt{N} u_k(i) f(i)
    \end{equation}
  \end{block}  

  If a kernel $f$ is localized around 0 in the graph spectral domain, then
  $\overline{M_kg}$ is localized around $\lambda_k$.
\end{frame}

\begin{frame}
  \frametitle{Dilation}

  \begin{block}{Classical dilation}
    \begin{equation}
      \begin{split}
        \text{Time domain: } (D_s f)(t) = \frac{1}{s} f \bigg( \frac{t}{s} \bigg) \\
        \text{Frequency domain: } \overline{D_s f}(\xi) = \hat{f}(s \xi)
      \end{split}
    \end{equation}
  \end{block}  

  Replace the frequency $\xi$ with an eigenvalue of the Laplacian.
  
  \begin{block}{Graph dilation}
    \begin{equation}
      (D_sf)(\lambda) = \hat{f} (s \lambda)
    \end{equation}
  \end{block}  
\end{frame}

\begin{frame}
  \frametitle{Graph coarsening, downsampling and reduction}

  \begin{itemize}
  \item For bipartite graphs, one can recursively downsample by a factor of two
  \item Downsampling based on diffusion distances
  \item Greedy seed selection algorithm
  \item Recursive spectral bisection
  \item Minimize number of edges connecting two vertices in a downsampled subset
  \end{itemize}
\end{frame}

\begin{frame}
  \frametitle{Localized multiscale transforms}

  Measuring the spread of graph signals in both time and frequency domains:

  \begin{block}{Spatial spread of a signal f around a center vertex i}
    \begin{equation}
      \Delta_{G,i}^2(f) = \frac{1}{|| f ||_2^2} \sum_{j \in V} [d_G(i, j)]^2 [f(j)]^2
    \end{equation}
  \end{block}  

  \begin{itemize}
  \item Where $d_G(i, \cdot)$ is the geodesic distance function.
  \item $[f(j)]^2 / || f ||^2$ represents the pmf of signal $f$.
  \item $\Delta_{G, i}^2$ is the variance of the geodesic distance function at
    node i.
  \end{itemize}
\end{frame}

\begin{frame}
  \frametitle{Localized multiscale transforms}
  
  The spatial and spectral spreads can thus both be characterized:

  \begin{block}{Total spatial spread of a graph signal}
    \begin{equation}
      \Delta_{G}^2(f) = min_{i \in V} \{ \Delta_{G, i} (f) \}
    \end{equation}
  \end{block}  
  \begin{block}{Total spectral spread of a graph signal}
    \begin{equation}
      \Delta_{\sigma}^2(f) = min_{\mu \in \mathcal{R}} \bigg\{ \frac{1}{|| f ||_2^2} \sum_{\lambda \in \sigma(L)} [\sqrt{\lambda} - \sqrt{\mu}]^2 [\hat{f}(\lambda)]^2 \bigg\}
    \end{equation}
  \end{block}  
 \end{frame} 
  


%------------------------------------------------

\begin{frame}
\frametitle{Figure}
Uncomment the code on this slide to include your own image from the same directory as the template .TeX file.
%\begin{figure}
%\includegraphics[width=0.8\linewidth]{test}
%\end{figure}
\end{frame}

%------------------------------------------------

\begin{frame}[fragile] % Need to use the fragile option when verbatim is used in the slide
\frametitle{Citation}
An example of the \verb|\cite| command to cite within the presentation:\\~

This statement requires citation \cite{p1}.
\end{frame}

%------------------------------------------------

\section{Strengths and weaknesses}

\section{Reproducibility}

\section{Extensions and applications}


\begin{frame}
\frametitle{References}
\footnotesize{
\begin{thebibliography}{99} % Beamer does not support BibTeX so references must be inserted manually as below
\bibitem[Smith, 2012]{p1} John Smith (2012)
\newblock Title of the publication
\newblock \emph{Journal Name} 12(3), 45 -- 678.
\end{thebibliography}
}
\end{frame}


\begin{frame}
\frametitle{Appendix}

For an infinite square lattice grid, it can be shown that the Graph Laplacian
corresponds to the continuous Laplacian as $\epsilon \rightarrow 0$:

  \begin{block}{Equivalence between Continous and Graph Laplacians}
    \begin{equation}
      \frac{\partial^2 F}{\partial x^2} = \lim_{\epsilon \rightarrow 0} \frac{[F(x + \epsilon) - F(x)] + [F(x - \epsilon) - F(x)]}{\epsilon^2}
    \end{equation}
  \end{block}

\end{frame}


%------------------------------------------------

\begin{frame}
\Huge{\centerline{The End}}
\end{frame}

%----------------------------------------------------------------------------------------

\end{document} 